\documentclass[12pt]{article}
\usepackage[utf8]{inputenc}
\usepackage{lipsum}
\usepackage{titling}
\usepackage{hyperref}
\usepackage{natbib}
\hypersetup{
    colorlinks=true,
    linkcolor=blue,
    filecolor=magenta,      
    urlcolor=cyan,
    pdftitle={Overleaf Example},
    pdfpagemode=FullScreen,
    }

\urlstyle{same}

\title{The Pollution Impacts of Low Reporting Standards}
\author{Aaron Watt}
\date{February 2022}
\setlength{\droptitle}{-10em}

% REVIEW COUNT -- goal is 100 by April
% increase the count each time the intro is reviewed
% 1


% Helpful docs:
% CONVERSABLE ECONOMIST: Writing the Intro to Your Economics Research Paper (Timothy Taylor)
% Bellemare, “How to Write Applied Papers in Economics.”
% McCloskey, Economical Writing.


% Task: Submission of your complete paper. Typically a complete paper will include an introduction, a model, an econometric equation, a description of the data, discussion of results, and a conclusion. The paper should be no more than 30 pages using a 12 pt font, 1-inch margins, including tables, graphs and reference, and with the core text double-spaced.


\begin{document}

\maketitle

%=====================================================================
\begin{abstract}
%=====================================================================
% Typically, it is possible to write a solid draft of your abstract by keeping only the first sentence of the hook, research question, and value added sections of your introduction, and by polishing up the resulting paragraph some.
% Except for the requisite terminology (e.g., randomized controlled trial, difference-in-differences, regression discontinuity), your abstract should be intelligible to any smart, college-educated person who is not an economist. This is especially true for an applied economics paper. After all, we are writing about real-world phenomena that are of interest to policy makers or business managers, so your abstract should be intelligible to someone with a master’s degree in public policy or in business administration, depending on what you are doing. Do not make the mistake of confusing lack of intelligibility with intellectual rigor; this is economics, not French postmodern philosophy.
% If your title is not repellent, and if your abstract is intelligible to people who are not experts in your field and to people in other disciplines, you have just expanded the scope of your citations tenfold, because whether one likes it or not, a lot of people cite stuff they have only read the abstract of.


% Research Question


% Proposed Model


% Dataset


% Estimation and testing plan


\vspace{2em}
\end{abstract}





% Research Question
% How much do prediction errors matter in pollution regulation? Does incorporating prediction errors of machine-learning-produced pollution data affect the policy categorization of areas without a pollution monitor?


%=====================================================================
\section{Introduction}
%=====================================================================

  
% Something about measurement error in environmental economics... robustness-to-measurement-error replication study, focusing on omitted measurement error from scientifically measured data (like temperature, elevation, pollution concentration, satellite approximated surface characteristics, etc).


% =====================================
%               1. HOOK (1-2)
% =====================================
% A good introduction starts with a good “hook,” i.e., something that grabs the reader’s attention and makes her want to keep reading. Here, the closer one can get to the reader, the better. Likewise, the broader one can go, the better. Bad hooks tend to appeal to the literature: “A long literature in economics has looked at ...” If that is the case, do you really want to make it any longer? Good hooks tend to relate to the real world: A lot of the food we buy at the grocery store is grown in the context of long value chains. What does the first link in that value chain look like? What does participating in those value chains do for the people who actually grow the food we eat? The hook should be one or two paragraphs long.

% Attract the reader’s interest by telling them that this paper relates to something interesting. What makes a topic interesting? Some combination of the following attributes makes Y something worth looking at.
% Y matters: When Y rises or falls, people are hurt or helped.
% Y is puzzling: it defies easy explanation.
% Y is controversial: some argue one thing while other say another.
% Y is big (like the service sector) or common (like traffic jams).

% Things to avoid:
% The bait and switch : promising an interesting topic but delivering something else, in particular, something boring.
% “all my friends are doing it” : presenting no other motivation for a topic than that other people have written papers on it.





% =====================================
%       2. RESEARCH QUESTION (1)
% =====================================
% After hooking the reader in and setting the stage, it is time to state your research question as clearly as possible. I like to do so by stating my actual research question as the first sentence of this part of my introductions. “What is the impact of participation in contract farming on the welfare of those who participate?” The clearer this is stated, the better, because the fewer are the occasions for the reader to be disappointed. This should be one paragraph long.

% Tell the reader what this paper actually does. Think of this as the point in a trial where having detailed the crime, you now identify a perpetrator and promise to provide a persuasive case. The reader should have an idea of a clean research question that will have a more or less satisfactory answer by the end of the paper. Examples follow below. The question may take two paragraphs. At the end of the first (2nd paragraph of the paper) or possibly beginning of the second (3rd paragraph overall) you should have the “This paper addresses the question” sentence.

% Research question

% Explanation




% =====================================
%           3. ANTECEDENTS
% =====================================
% After stating your research question, it is time to relate it and what you are doing to the existing literature. Here, relate your work to the five to ten closest studies (the closer to five, the better) in the literature. What the relevant literature -- the antecedents—is will obviously depend on the question at hand. If you are lucky enough to work in a literature that has seen a lot of activity, you may have a hard time narrowing it down, and you will need to judiciously pick the five to ten closest studies. If you are working on a problem that no one has really looked at, or that no one has looked at in a long time, you might have to go back in time a bit further or expand your parameters for what counts as antecedents. Here, what counts is to tell a bit of a story; no one wants to read a bland enumeration of studies: “Johnson (2011) found this. Wang (2012) found that. Smith (2013) found something else. Patel (2015) found something else altogether.” For every topic, the intellectual history of that topic can be told in an interesting way.

% Relate your work to the five closest studies in the literature
% tell a bit of a story




% =====================================
%           4. VALUE ADDED
% =====================================
% This is where you need to shine. What is your contribution? How does your paper change people’s priors about your topic? Ideally, your paper will have three contributions. For instance, you may be improving on the internal validity front for the question you are looking at by having a better identification strategy. You may also be improving on the external validity front by having data that cover a broader swath of the real world; or you may be performing a mediation analysis that allows identifying what mechanism m the treatment variable D operates through in causing changes in y. Lastly, you may also be bringing a small methodological improvement to the table. But even papers with one contribution deserve to be published, provided that contribution is important enough.


% Contribution 1


% Contribution 2


% Contribution 3





% =====================================
%           5. ROAD MAP
% =====================================
% Lastly, you should provide your reader with a roadmap to your paper. This section usually starts with “The remainder of this article is organized as follows,” and it lists section and what they do in order. So for a typical paper, it would go: “The remainder of this paper is organized as follows. Section 2 presents the theoretical framework used to study the research question and derives this paper’s core testable prediction. In section 3, the empirical framework is presented, first by discussing the estimation strategy, and then by discussing the identification strategy. Section 4 presents the data and discusses some summary statistics. In section 5, the empirical results are presented and discussed, followed by a battery of robustness checks and a discussion of the limitations of the results. Section 6 concludes with policy recommendations and suggestions for future research.” I have seen some economists on social media state that they have gotten papers rejected for many reasons, but never for want of a roadmap section. Fair enough. In most cases, however, it is simply easier to include such a roadmap section and delete it at a reviewer’s request than to not have one and have to write one when asked to revise and resubmit a paper, not to mention the fact that some readers will simply expect there to be a roadmap, since the majority of applied economics articles include them. Anything that signals that you know what the unspoken rules and norms of the profession are is a good thing for your article’s chances of getting published.

% Section 1.b (optional): background (not lit review)
% Section 2: theoretical framework, testable prediction
% Section 3: empirical framework: estimation strategy, identification strategy
% Section 4: data, summary stats
% Section 5: empirical results, discussion, robustness checks, limitations of the results
% Section 6: policy recommendations, future research


The remainder of this article is organized as follows. Section \ref{background} briefly reviews the history of air quality standard in the US and some key details of current regulations. Section \ref{theoretical} explains the theoretical framework applied to the problem of measuring pollution at specific points in space. Section \ref{data} then discusses the data used and section \ref{emprical} describes the empirical framework that will be applied to estimate the missing pollution and resulting policy outcomes. Section \ref{results} reviews the results of the empirical study and discusses the implications. Section \ref{conclusion}


\section{Background} \label{background}


\section{Theoretical Framework} \label{theoretical}


\section{Data and Descriptive Statistics} \label{data}


\section{Empirical Framework} \label{emprical}

\section{Results and Discussion} \label{results}

\section{Summary and Concluding Remarks} \label{conclusion}









\newpage
\bibliographystyle{chicago}
\bibliography{references}
% Citation command	Output
% \citet{goossens93}      Goossens et al. (1993)
% \citep{goossens93}      (Goossens et al., 1993)	
% 
% \citet*{goossens93}     Goossens, Mittlebach, and Samarin (1993)
% \citep*{goossens93}     (Goossens, Mittlebach, and Samarin, 1993)
%
% \citeauthor{goossens93}    Goossens et al.
% \citeauthor*{goossens93}   Goossens, Mittlebach, and Samarin

% \citeyear{goossens93}      1993
% \citeyearpar{goossens93}  (1993)

% \citealt{goossens93}     Goossens et al. 1993
% \citealp{goossens93}     Goossens et al., 1993
% \citetext{priv.\ comm.}	(priv. comm.)

\section{Appendix}



\end{document}
