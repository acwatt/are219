%\citep{91st_us_congress_clean_1970}
Amid growing public concern about air quality and pollution, the United States Congress passed the Clean Air Act of 1963 (CAA). Later additions to the CAA, the Clean Air Amendments of 1970, granted the Environmental Protection Agency (EPA) the regulatory authority to create and enforce air quality standards in the US. One major way air quality is regulated is through the National Ambient Air Quality Standards (NAAQS), which set concentration thresholds for a list of different ``criteria'' pollutants
\citep{91stuscongressCleanAirAmendments1970}. The EPA has since been in charge of setting and updating the NAAQS and require states to submit plans to bring their air quality to within NAAQS limits. An important aspect of enforcing the NAAQS is measuring criteria pollutants across the US by requiring states to install pollution monitoring stations in areas of questionable air quality. Because these monitoring stations are used for potentially costly enforcement, the equipment within each station must abide by specific regulations and are relatively costly to install and run.

Over the last decade, commercially available scientific equipment in measuring various air pollutants has evolved. There is now relatively cheap\footnote{e.g., a PurpleAir outdoor air quality sensor is about \$250 to purchase with little upkeep from the end user, compared to roughly \$100,000-200,000 to install EPA regulation-grade criteria pollutant monitors and trained staff to upkeep and record measurements. The cost alone is not a good comparison because the EPA monitors use different technology that is known to be more accurate across a wider range of pollution concentrations, have a better sense the sensor error, and measure more pollutants than the PurpleAir monitors. For the purposes of this analysis, PurpleAir monitors should be seen as a compliment to EPA monitors, not a potential replacement.} equipment available to measure particulate matter (one of the criteria pollutants that regulated by the NAAQS). Specifically, the PurpleAir company produces devices that can measure particulate matter that has a diameter of less than 2.5 micrometers (designated as PM2.5).\footnote{PurpleAir devices can measure a few other criteria pollutants (namely ozone and PM10) but the comparability of the PM2.5 measurements between PurpleAir and EPA monitors are currently better understood.} PurpleAir is of particular interest because they have built an opt-out mechanism for end-users to allow their ambient air quality data to be stored in the cloud. They also provide multiple ways for researchers and the general public to use this crowd-sourced air quality data.

This paper is primarily concerned with the minimum reporting requirement. As with many federal regulations, there are many ways that states or emitters can cleverly navigate the rules to emit more than they are meant to according to the spirit of the regulation. One way of navigating the CAA regulations is through the choice of what data to report. The EPA currently requires a minimum threshold of air quality data to be reported -- for PM 2.5, 75\% of daily measurements need to be reported, and each day must have 75\% of hours reported. That leaves many choices of which hours to turn the monitor off for cleaning, calibration, or other reasons. I wish to understand how these timing decisions are affecting the distribution of reported data -- specifically how it might be affecting a statistic of that distribution: the design value.