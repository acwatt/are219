\documentclass{beamer}
\usepackage[utf8]{inputenc}
\setbeamertemplate{navigation symbols}{}

\usetheme{Madrid}
\usecolortheme{default}


%------------------------------------------------------------
%This block of code defines the information to appear in the
%Title page
\title[Missing Pollution Data] %optional
{Filling in the Gaps: Using Consumer Products to Replace Missing Pollution Data}

% \subtitle{A short story}

\author[Watt, Aaron] % (optional)
{A.~C.~Watt\inst{1}}

\institute[UCB] % (optional)
{
  \inst{1}%
  Agricultural \& Resource Economics\\
  University of California, Berkeley
}

\date[SYP 2022] % (optional)
{Second Year Paper, March 2022}

%---------------------------------------------------------



%---------------------------------------------------------
% Introduction

\AtBeginSection[]
{
  \begin{frame}
    \frametitle{Table of Contents}
    \tableofcontents[currentsection]
  \end{frame}
}
%------------------------------------------------------------


\begin{document}

%The next statement creates the title page.
\frame{\titlepage}


%---------------------------------------------------------
%This block of code is for the table of contents after
%the title page
\begin{frame}
\frametitle{State of project}
\textbf{Data (2016-2021)} 

\begin{itemize}
    \item have: 15 CA EPA sites
    \item have: all hourly data for PA sites within 50 miles of all 15 EPA sites
    \item want: all US PA data
    \item can before April 11: get 10-50 more EPA sites
\end{itemize}
\vspace{2em}
\textbf{Analysis}
\begin{itemize}
\item have: Inverse distance weighted mean of PA
\item have: OLS to predict EPA with PA, give prediction intervals\\
$EPA_t = \beta_0 + \beta_1 PA^{IDW}_t + \varepsilon_t
$
\item want: prediction (1): more complicated OLS with windspeed and direction
\item prediction (2): machine learning model
\item can before April 11: probably windspeed and direction model
\end{itemize}

\end{frame}
%---------------------------------------------------------



%---------------------------------------------------------
\begin{frame}
\frametitle{Data Use Agreement}

\begin{itemize}
    \item Gave Adrian Dybwad draft person DUA
    \item Adrian said legal folks ask if we could do this at the University level
    \item I asked if we could do a temporary DUA while we get the university level one figured out
\end{itemize}

What are the first steps to getting a university level DUA?

\end{frame}

%---------------------------------------------------------


% %---------------------------------------------------------
% %Example of the \pause command
% \begin{frame}
% \frametitle{National Ambient Air Quality Standards for PM2.5}
% Two standards for assessing compliance for ambient PM2.5 concentrations at the location of a monitor:
% \vspace{2em}

% \textbf{Daily Design Value} (average)
% \begin{itemize}
%     \item[$\Rightarrow$] 3-year rolling average of 1-year average of daily averages
%     \item[$\Rightarrow$] standard currently set at 15.0 $\mu$g/m$^3$
% \end{itemize}
% \vspace{1em}

% \textbf{24-Hour Design Value} (98$^{th}$ percentile)
% \begin{itemize}
%     \item[$\Rightarrow$] 3-year rolling average of 1-year 98$^{th}$ percentile of daily averages
%     \item[$\Rightarrow$] standard currently set at 35 $\mu$g/m$^3$
% \end{itemize}
% \end{frame}
% %---------------------------------------------------------


% %---------------------------------------------------------
% %Highlighting text
% \begin{frame}
% \frametitle{NAAQS Completeness Criteria for daily monitors}

% \textbf{Valid day:}
% minimum of 75\% hours reported  (18 hours)

% \vspace{2em}
% \textbf{Valid quarter:}
% minimum of 75\% valid days  (22-23 days\footnote{fewer for monitors that report less frequent observations})

% \vspace{2em}
% \textbf{Valid quarterly design value:}
% every in 3-year period
% (12 quarters)

% \end{frame}
% %---------------------------------------------------------


%---------------------------------------------------------
%Observed EPA Completeness
\begin{frame}
\frametitle{Observed Completeness of NAAQS Monitors in sample}
\includegraphics[width=\textwidth]{output/figures/epa_vs_pa/completeness/completeness_epa_pa.png}
\end{frame}
%---------------------------------------------------------


%---------------------------------------------------------
%Purple Air Outdoor Monitors
\begin{frame}
\frametitle{PurpleAir Outdoor Monitors}
\hspace{1em}
PurpleAir Sensors in California 
\hspace{4em}
LA Site Example \\

\includegraphics[width=0.49\textwidth]{output/figures/pa/all_ca_and_15_pa_monitors.png}
\includegraphics[width=0.49\textwidth]{output/figures/concentric_ranges/county-037_site-4004_epa-pa-concentric-ranges.png}
\end{frame}
%---------------------------------------------------------


%---------------------------------------------------------
%Alternate measure of ambient air quality
\begin{frame}
\frametitle{Alternate measure of ambient PM2.5 Concentration}

\hspace{7em}
\includegraphics[width=0.5\textwidth]{output/figures/diagrams/IDW_diagram.png}
\\
\textbf{Inverse-distance Weighted Average Ambient PM2.5}

\begin{equation*}
    PA^{IDW}_t= \sum\limits_{j=1}^{J_t} \dfrac{\frac{1}{d_j}\cdot PA_{j,t}}{\sum\limits_j^{J_t}\frac{1}{d_j}}
    = \sum\limits_{j=1}^{J^t} w_{j,t}\cdot PA_{j,t}
\end{equation*}
\begin{itemize}
    \item $J_t$ = active PurpleAir sensors around the NAAQS monitor at time $t$
\end{itemize}

\end{frame}
%---------------------------------------------------------


%---------------------------------------------------------
%Predicting Missing EPA Data
\begin{frame}
\frametitle{Predicting Missing EPA Data}
\begin{equation*}
EPA_t = \beta_0 + \beta_1 PA^{IDW}_t + \varepsilon_t
\end{equation*}

\begin{table}[!htbp] \centering
  \caption{Reported NAAQS Monitor PM2.5 (site 037-4004)}
  \label{tab:reg_037-4004}
\begin{tabular}{@{\extracolsep{5pt}}lcc}
\\[-1.8ex]\hline
\hline \\[-1.8ex]
\\[-1.8ex] & (1) & (2) \\
\hline \\[-1.8ex]
 intercept & & 6.924$^{***}$ \\
  & & (0.076) \\
 PurpleAir IDW Average & 0.741$^{***}$ & 0.444$^{***}$ \\
  & (0.003) & (0.004) \\
\hline \\[-1.8ex]
 Observations & 36,813 & 36,813 \\
 $R^2$ & 0.658 & 0.240 \\
 F Statistic & 70924.412$^{***}$  & 11642.169$^{***}$  \\
\hline
\hline \\[-1.8ex]
& \multicolumn{2}{r}{$^{*}$p$<$0.1; $^{**}$p$<$0.05; $^{***}$p$<$0.01} \\
\end{tabular}
\end{table}
\end{frame}
%---------------------------------------------------------


%---------------------------------------------------------
%Design Values for an EPA site
\begin{frame}
\frametitle{Calculating Design Values for an EPA Site}

\def\dvp{\text{DV}_p}
\def\dvpp{\widetilde{\text{DV}}_p}

\textbf{Pseudo Design Values}\\[0.5em]
$\dvp$ = $p$ design value for quarter, calculated using \textbf{reported} PM2.5 data from EPA monitor, $\forall p \in \{\text{Daily, 24-Hour}\}$
\\[1em]

\textbf{Imputed Design Values}\\[0.5em]
$\dvpp$ = $p$ design value for quarter, calculated using \textbf{imputed} PM2.5 data from EPA monitor and PA sensors
\\[2em]
\textbf{Bias in Design Values}
\begin{equation*}
    \text{bias from missing data in }\dvp \approx \dvpp - \dvp
\end{equation*}

\end{frame}
%---------------------------------------------------------


%---------------------------------------------------------
%Results: Sample EPA Sites
\begin{frame}
\frametitle{Results for Daily Design Value: Sample EPA Sites}
\includegraphics[width=\textwidth]{output/figures/final_results/DV_annual_plot_all_test_sites.png}

{\small Shaded regions are 95\% confidence intervals from interpolating the data.}
\end{frame}
%---------------------------------------------------------


%---------------------------------------------------------
%Results: Fresno
\begin{frame}
\frametitle{Results for Daily Design Value: Fresno}


\includegraphics[width=\textwidth]{output/figures/final_results/DV_annual_plot_site_019-0500.png}


{\small Shaded regions are 95\% confidence intervals from interpolating the data.}
\end{frame}
%---------------------------------------------------------


%---------------------------------------------------------
%Results: Sample EPA Sites
\begin{frame}
\frametitle{Results for 24-hour Design Value: Sample EPA Sites}
\includegraphics[width=\textwidth]{output/figures/final_results/DV_hour_plot_all_test_sites.png}

{\small Shaded regions are 95\% confidence intervals from interpolating the data.}
\end{frame}
%---------------------------------------------------------


%---------------------------------------------------------
%Results: Fresno
\begin{frame}
\frametitle{Results for 24-hour Design Value: Fresno}


\includegraphics[width=\textwidth]{output/figures/final_results/DV_hour_plot_site_019-0500.png}

{\small Shaded regions are 95\% confidence intervals from interpolating the data.}
\end{frame}
%---------------------------------------------------------


%---------------------------------------------------------
%Conclusions & Discussion
\begin{frame}
\frametitle{Conclusions \& Discussion}
\begin{enumerate}
\setlength\itemsep{2em}
    \item Most tested sites show little evidence of bias from missing data, but one has large, meaningful bias
    \item Even one site can affect millions of people due to the sparsity of monitoring sites
    \item As high-pollution locations continue to reduce pollution, this bias may play an important role in determining NAAQS compliance
    \item Underlines importance of expanding the monitor network or exploring alternative measures of ambient air quality
\end{enumerate}
\end{frame}
%---------------------------------------------------------


%---------------------------------------------------------
%Future Work
\begin{frame}
\frametitle{Future Work}

\begin{itemize}
\setlength\itemsep{2em}
    \item Optimal regulation of ambient pollution under monitor expense-accuracy tradeoff
    \item Expand test to rest of US monitors
    \item Explore spatial distribution of air quality in unmonitored locations
\end{itemize}

\end{frame}
%---------------------------------------------------------


%---------------------------------------------------------
% Purple Air EPA Correction
\begin{frame}
\frametitle{Appendix: Correction of PurpleAir Readings}
$$
\widetilde{PA}_{j,t}=\begin{cases}
			0.52*PA_{j,t} - 0.086*H_{j,t} + 5.75, & \text{if $PA_{j,t} \leq 343 \mu$g/m$^3$}\\
            0.46*PA_{j,t} + 0.(3.93e-4)PA_{j,t}^2 + 2.97, & \text{otherwise}
		 \end{cases}
$$
\begin{itemize}
    \item $PA_{j,t}$ = ambient PM2.5 measured by PurpleAir sensor $j$ at time $t$
    \item $H_{j,t}\in[0,1]$ is the relative humidity
\end{itemize}
\end{frame}
%---------------------------------------------------------


%---------------------------------------------------------
% Future Prediction Equation
\begin{frame}
\frametitle{Appendix: Better Prediction of EPA PM2.5}

$$
\widetilde{PA}_{j,t}=\begin{cases}
			0.52*PA_{j,t} - 0.086*H_{j,t} + 5.75, & \text{if $PA_{j,t} \leq 343 \mu$g/m$^3$}\\
            0.46*PA_{j,t} + 0.(3.93e-4)PA_{j,t}^2 + 2.97, & \text{otherwise}
		 \end{cases}
$$
\begin{itemize}
    \item $PA_{j,t}$ = ambient PM2.5 measured by PurpleAir sensor $j$ at time $t$
    \item $H_{j,t}\in[0,1]$ is the relative humidity
\end{itemize}
\end{frame}
%---------------------------------------------------------



%---------------------------------------------------------
% Fresno Site Data
\begin{frame}
\frametitle{Appendix: Fresno Site Data}

\hspace{5em}
\includegraphics[width=0.8\textwidth]{pics/appendix/site_plots/site-019-0500_epa-pa-hourly-plot.png}
\end{frame}
%---------------------------------------------------------


%---------------------------------------------------------
%Extra Tables
\begin{frame}
\frametitle{Appendix: Fresno, California DV Bias Table}
\vspace{-1em}
\begin{table}[]
\caption{Design Value Comparison for Fresno, CA. (95\% CI Bounds)}
\label{tab:fresno_dvs}
\resizebox{\textwidth}{!}{%
\begin{tabular}{@{}l|lll|lll@{}}
Year-Quarter & \begin{tabular}[c]{@{}l@{}}Annual DV\\ Difference\end{tabular} & \begin{tabular}[c]{@{}l@{}}Upper \\ Bound\end{tabular} & \begin{tabular}[c]{@{}l@{}}Lower\\ Bound\end{tabular} & \begin{tabular}[c]{@{}l@{}}Hour DV \\ Difference\end{tabular} & \begin{tabular}[c]{@{}l@{}}Upper \\ Bound\end{tabular} & \begin{tabular}[c]{@{}l@{}}Lower \\ Bound\end{tabular} \\ \midrule
2018-4 & Invalid DV &  &  & Invalid DV &  &  \\
2019-1 & -0.005 & 0.133 & -0.143 & 0.000 & 0.252 & 0.000 \\
2019-2 & -0.003 & 0.141 & -0.147 & 0.000 & 0.252 & 0.000 \\
2019-3 & 0.015 & 0.170 & -0.139 & 0.058 & 2.202 & 0.000 \\
2019-4 & 0.002 & 0.190 & -0.185 & 0.000 & 1.460 & -0.024 \\
2020-1 & -0.012 & 0.182 & -0.207 & 0.000 & 0.335 & 0.000 \\
2020-2 & -0.010 & 0.191 & -0.211 & 0.000 & 0.376 & 0.000 \\
2020-3 & 0.679 & 0.954 & 0.403 & 8.718 & 11.704 & 8.556 \\
2020-4 & 0.647 & 1.006 & 0.288 & 5.979 & 8.281 & 5.851 \\
2021-1 & 0.564 & 1.036 & 0.091 & 3.007 & 4.184 & 2.903 \\
2021-2 & 0.533 & 1.024 & 0.042 & 3.007 & 4.225 & 2.903 \\
2021-3 & 0.630 & 1.129 & 0.132 & 7.607 & 10.557 & 7.444
\end{tabular}%
}
\end{table}
\end{frame}
%---------------------------------------------------------












\end{document}




%    Examples

%%%%%%%%%%%%%%%%%% multi-slide text
\begin{frame}
In this slide \pause

the text will be partially visible \pause

And finally everything will be there
\end{frame}





%%%%%%%%%%%%%%%%%% highlighting text and theorem boxes
\begin{frame}
\frametitle{NAAQS Completeness Criteria}

In this slide, some important text will be
\alert{highlighted} because it's important.
Please, don't abuse it.

\begin{block}{Remark}
Sample text
\end{block}

\begin{alertblock}{Important theorem}
Sample text in red box
\end{alertblock}

\begin{examples}
Sample text in green box. The title of the block is ``Examples".
\end{examples}
\end{frame}








%Two columns
\begin{frame}
\frametitle{Two-column slide}

\begin{columns}

\column{0.5\textwidth}
This is a text in first column.
$$E=mc^2$$
\begin{itemize}
\item First item
\item Second item
\end{itemize}

\column{0.5\textwidth}
This text will be in the second column
and on a second tought this is a nice looking
layout in some cases.
\end{columns}
\end{frame}








